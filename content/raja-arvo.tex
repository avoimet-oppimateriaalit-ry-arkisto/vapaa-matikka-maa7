\chapter{Raja-arvo}

Rationaalifunktio $f: \rr \backslash \{0 \} \to \rr$, $f(x) = \frac{x^2}{x}$ voidaan määrittelyjoukossaan sieventää muotoon $f(x) = x$. Mitä jos $x = 0$? Lauseketta $\frac{0}{0}$ ei ole määritelty, joten funktiota ei ole mielekästä jatkaa nollassa samalla lausekkeella. Funktion kuvaaja näyttää kuitenkin suoralta.

KUVA

Itse pistettä $x = 0$ lukuun ottamatta funktio käyttäytyy siististi, ja tuntuu luontevalta määritellä, että $f(x) = 0$. $x$:n lähestyessä nollaa myös $f(x)$ näyttää lähestyvän rajatta nollaa.

Lukiossa ja peruskoulussa on totuttu ns. \termi{jatkuva funktio}{jatkuviin funktioihin}, eli käytännössä funktioihin, joiden kuvaaja on katkeamaton. Jatkuvilla funktioilla on monia hyödyllisiä ominaisuuksia, mutta myös \termi{epäjatkuva funktio}{epäjatkuvat funktiot}, funktiot jotka ovat "katkonaisia" näyttävät käyttäytyvän välillä mukavasti.

Esimerkin avulla huomattiin, että vaikka $f(x)$ ei ollut määritelty origossa, näyttivät funktion arvot lähestyvän nollaa.  Tällöin sanotaan, että funktiolla on \termi{raja-arvo}{raja-arvo} nollassa, nolla.

Raja-arvo on erittäin tärkeä käsite matematiikassa. Intuitiivisesti sillä tarkoitetaan arvoa, jota funktion arvot lähestyvät jossain pisteessä. Usein tutkittavan funktion arvoa voi olla vaikea analysoida, mutta pisteen lähistöllä funktio käyttäytyy mukavasti. Raja-arvo on ikään kuin tapa testata, miten funktio suhtautuu muutokseen.

Miten funktio  $f: \rr \backslash \{0 \} \to \rr$, $f(x) = \frac{1}{x}$ käyttäytyy nollassa? Funktiota ei ole mielekästä jatkaa nollassa, mutta nyt luonnollista täytepistettä ei näytä löytyvän.

KUVA

Kun $x$ on positiivinen ja lähestyy nollaa, funktion arvot kasvavat äärettömyyksiin. Ääretön ei kuitenkaan sovi funktion arvoksi. Toisaalta kun $x$ on negatiivinen funktion arvot näyttävät pienenevän rajatta. Tässä tapauksessa funktiolla ei ole raja-arvoa nollassa. Lauseke $\frac{1}{0}$ ei tarjoa oikeastaan paljoakaan, mutta tutkimalla funktion arvoa nollan lähellä voidaan funktiota kuvata jo aivan uudella tavalla.

%semiformaali määritelmä

% jotain hyvin epäjatkuvia funktioita
% jatkuvuuden käsite

\laatikko[Raja-arvosäännöt]{
	\begin{description}
		\item[Summan raja-arvo] $\lim\limits_{x \to a} (f+g)(x) = \lim\limits_{x \to a} f(x) + \lim\limits_{x \to a} g(x)$
		\item[Tulon raja-arvo] $\lim\limits_{x \to a} (fg)(x) = \lim\limits_{x \to a} f(x) \cdot \lim\limits_{x \to a} g(x)$
		\item[Osamäärän raja-arvo] $\lim\limits_{x \to a} (\frac{f}{g})(x) = \frac{\lim\limits_{x \to a} f(x)}{\lim\limits_{x \to a} g(x)}$
	\end{description}
}

\begin{tehtavasivu}

\end{tehtavasivu}
