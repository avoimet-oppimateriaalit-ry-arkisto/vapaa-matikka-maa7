\chapter{Raja-arvo}

\section{Johdanto}

Rationaalifunktio $f: \rr \backslash \{0 \} \to \rr$, $f(x) = \frac{x^2}{x}$ voidaan määrittelyjoukossaan sieventää muotoon $f(x) = x$. Entä jos $x = 0$? Lauseketta $\frac{0}{0}$ ei ole määritelty, joten funktiota ei ole mielekästä jatkaa nollassa samalla lausekkeella. Funktion kuvaaja näyttää kuitenkin suoralta.

\begin{luoKuva}{raja-arvo_esimerkki}
	kuvaaja.pohja(-2, 2, -2, 2, leveys=5)

	kuvaaja.piirra("x")
	
	geom.ympyra((0,0), 0.04)
	vari("white")
	geom.ympyra((0,0), 0.015)	
\end{luoKuva}

\begin{center}
	\naytaKuva{raja-arvo_esimerkki}
\end{center}

Itse pistettä $x = 0$ lukuun ottamatta funktio käyttäytyy siististi, ja tuntuu luontevalta määritellä, että $f(x) = 0$. $x$:n lähestyessä nollaa myös $f(x)$ näyttää lähestyvän rajatta nollaa.

Lukiossa ja peruskoulussa on totuttu ns. \termi{jatkuva funktio}{jatkuviin funktioihin}, eli käytännössä funktioihin, joiden kuvaaja on katkeamaton. Jatkuvilla funktioilla on monia hyödyllisiä ominaisuuksia, mutta myös \termi{epäjatkuva funktio}{epäjatkuvat funktiot}, funktiot jotka ovat ''katkonaisia'' näyttävät käyttäytyvän toisinaan mukavasti.

Esimerkin avulla huomattiin, että vaikka $f(x)$ ei ollut määritelty origossa, funktion arvot lähestyivät nollaa kummallakin puolella origoa. Tällöin sanotaan, että funktiolla on \termi{raja-arvo}{raja-arvo} jossakin pisteessä, tässä tapauksessa raja-arvo $0$ pisteessä $x=0$.

Raja-arvo on erittäin tärkeä käsite matematiikassa. Intuitiivisesti sillä tarkoitetaan arvoa, jota funktion arvot lähestyvät jossain pisteessä. %Usein tutkittavan funktion arvoa voi olla vaikea analysoida, mutta pisteen lähistöllä funktio käyttäytyy mukavasti. Raja-arvo on ikään kuin tapa testata, miten funktio suhtautuu muutokseen.

Miten funktio  $f: \rr \backslash \{0 \} \to \rr$, $f(x) = \frac{1}{x}$ käyttäytyy nollassa? Funktiota ei ole mielekästä jatkaa nollassa, mutta nyt luonnollista täytepistettä ei näytä löytyvän.

\begin{luoKuva}{raja-arvo_esimerkki}
    kuvaaja.pohja(-1, 10, -1, 10, leveys=5)
    kuvaaja.piirra("1/x", nimi = "$1/x$")
\end{luoKuva}

\begin{center}
	\naytaKuva{raja-arvo_esimerkki}
\end{center}

Kun $x$ on positiivinen ja lähestyy nollaa, funktion arvot kasvavat äärettömyyksiin. Ääretön ei kuitenkaan sovi funktion arvoksi. Toisaalta kun $x$ on negatiivinen funktion arvot näyttävät pienenevän rajatta. Tässä tapauksessa funktiolla ei ole raja-arvoa nollassa. Lauseke $\frac{1}{0}$ ei tarjoa oikeastaan paljoakaan, mutta tutkimalla funktion arvoa nollan lähellä voidaan funktiota kuvata jo aivan uudella tavalla.

\section{Määritelmä}

%semiformaali määritelmä

\section{Esimerkkejä}

% jotain hyvin epäjatkuvia funktioita

\section{Raja-arvosäännöt I}

\laatikko[Raja-arvosäännöt I]{
	\begin{description}
		\item[Summan raja-arvo] $\lim\limits_{x \to a} (f\pm g)(x) = \lim\limits_{x \to a} f(x) \pm  \lim\limits_{x \to a} g(x)$
		\item[Vakiolla kertomisen raja-arvo] $\lim\limits_{x \to a} (cf)(x) = c \cdot \lim\limits_{x \to a} f(x)$
	\end{description}
}

\begin{tehtavasivu}

\begin{tehtava}
Laske raja-arvot.
\alakohdat{
§$\lim\limits_{x \to 2} x^2+x^3$
§$\lim\limits_{x \to -6} -4+x^2$
§$\lim\limits_{x \to 0} x^6+x$
§$\lim\limits_{x \to 5} x^2+x+9$
}
\begin{vastaus}
 \alakohdat{§$12$§$32$§$0$§$39$}
\end{vastaus}
\end{tehtava}


\begin{tehtava}
Laske raja-arvot.
\alakohdat{
§$\lim\limits_{x \to 3} x^2-x$
§$\lim\limits_{x \to -1} x^-3-x^2$
§$\lim\limits_{x \to 0} x^4-x$
§$\lim\limits_{x \to 5} x-7$
}
\begin{vastaus}
 \alakohdat{§$6$§$-2$§$0$§$-2$}
\end{vastaus}
\end{tehtava}


\begin{tehtava}
Laske raja-arvot.
\alakohdat{
§$\lim\limits_{x \to 2} 8x^-2$
§$\lim\limits_{x \to 0} 2x^4$
§$\lim\limits_{x \to 2} -12x$
}
\begin{vastaus}
 \alakohdat{§$2$§$0$§$-24$}
\end{vastaus}
\end{tehtava}


\begin{tehtava}
Laske raja-arvot.
\alakohdat{
§$\lim\limits_{x \to 2} 6x^-1+3x-1$
§$\lim\limits_{x \to 0} 8x^3+3x^2-2x$
§$\lim\limits_{x \to 3} 2x^3+3x^2$
§$\lim\limits_{y \to -2} -12y⁻2$
}
\begin{vastaus}
 \alakohdat{§$8$§$0$§$81$§$-3$}
\end{vastaus}
\end{tehtava}

\end{tehtavasivu}

\section{Raja-arvosäännöt II}

\laatikko[Raja-arvosäännöt II]{
	\begin{description}
		\item[Tulon raja-arvo] $\lim\limits_{x \to a} (fg)(x) = \lim\limits_{x \to a} f(x) \cdot \lim\limits_{x \to a} g(x)$
		\item[Osamäärän raja-arvo] $\lim\limits_{x \to a} (\frac{f}{g})(x) = \frac{\lim\limits_{x \to a} f(x)}{\lim\limits_{x \to a} g(x)}$
	\end{description}
}

\begin{tehtavasivu}

\begin{tehtava}
Laske raja-arvot.
\alakohdat{
§$\lim\limits_{x \to 2} 3x\cdot2^x$
§$\lim\limits_{x \to 4} 2x\cdot\sqrt{x}$
§$\lim\limits_{x \to -1} (2x+1)(x+2)$
}
\end{tehtava}

\begin{tehtava}
Laske raja-arvot.
\alakohdat{
§$\lim\limits_{x \to 5} \frac{x^2-3}{x-3}$
§$\lim\limits_{x \to 4} \frac{6}{7-x}$
§$\lim\limits_{x \to -1} \frac{x^-3-x^2}{2x+1}$
\end{tehtava}

\end{tehtavasivu}



\section{Jatkuvuus}

\begin{tehtavasivu}

% tehtäviä epäjatkuvista funktioista

\begin{tehtava}
	Olkoon $f(x)=x^5$ ja $g(x)=x^3-2x$. Määritä raja-arvot.
	\begin{alakohdat}
		\alakohta{$\lim\limits_{x \to 2} f(x)$}
		\alakohta{$\lim\limits_{x \to 2} g(x)$}
		\alakohta{$\lim\limits_{x \to 2} (f+g)(x)$}
		\alakohta{$\lim\limits_{x \to 2} (fg)(x)$}		
	\end{alakohdat}
	\begin{vastaus}
		\begin{alakohdat}
			\alakohta{$32$}
			\alakohta{$4$}
			\alakohta{$36$}
			\alakohta{$128$}
		\end{alakohdat}
	\end{vastaus}
\end{tehtava}

\begin{tehtava}
	Funktion $f(x)$ raja-arvo pisteessä $x=1$ on $5$. Määritellään lisäksi $g(x) = x^2+3$. Mikä on funktion $h(x) = 4f(x) + 3g(x)$ raja-arvo pisteessä $x=1$?
	\begin{vastaus}
		$32$
	\end{vastaus}
\end{tehtava}

\end{tehtavasivu}
