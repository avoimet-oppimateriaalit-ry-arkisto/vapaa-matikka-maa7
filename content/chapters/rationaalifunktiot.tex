\chapter{Rationaalifunktiot}

\section{Rationaalifunktion määritelmä}

\termi{rationaalifunktio}{Rationaalifunktioiksi} kutsutaan funktioita, jotka voidaan esittää kahden
polynomifunktion osamääränä.

\begin{esimerkki}
	Rationaalifunktioita ovat esimerkiksi
	\begin{alakohdat}
		\alakohta{$x^2 = \frac{x^2}{1}$}
		\alakohta{$\frac1x$}
		\alakohta{$1+\frac2x+\frac1{x^2} = \frac{x^2+2x+1}{x^2}$}
		\alakohta{$\frac{x+1}{x-1}$}
	\end{alakohdat}
\end{esimerkki}

\section{Rationaalilausekkeiden sieventäminen}

\section{Rationaaliyhtälöt}

\begin{esimerkki}
	Ratkaise yhtälöt
	\begin{alakohdat}
		\alakohta{$\frac{2x+12}{x} = 8$}
		\alakohta{$\frac{2x+12}{x} = 2$}
		\alakohta{$\frac{2x+12}{x} = -4$}
	\end{alakohdat}
	\begin{esimratk}
		Ratkaistaan yleinen tapaus $\frac{2x+12}{x} = c$, missä $c \in \rr$.
		Todetaan ensin, että $x = 0$ ei voi olla yhtälön ratkaisu. Muulloin:
		\begin{align*}
			\frac{2x+12}{x} &= c \\
			2x+12 &= cx \\
			(2-c)x+12 &= 0 \\
			(2-c)x &= -12 \\
			x &= \frac{-12}{2-c} \\
			x &= \frac{12}{c-2}
		\end{align*}
		Havaitaan, että kun $c = 2$, ratkaisuja ei ole. Muut ratkaisut saadaan sijoittamalla.
	\end{esimratk}
	\begin{esimvast}
		\begin{alakohdat}
			\alakohta{$x = 2$}
			\alakohta{ei ratkaisuja}
			\alakohta{$x = -2$}
		\end{alakohdat}
	\end{esimvast}
\end{esimerkki}

\section{Rationaaliepäyhtälöt}

\begin{esimerkki}
	Ratkaise epäyhtälöt
	\begin{alakohdat}
		\alakohta{$\frac{2x+7}{x+10} < 1$}
		\alakohta{$\frac{2x+7}{x+10} > -1$}
		\alakohta{$|\frac{2x+7}{x+10}| < 1$.}
	\end{alakohdat}
	\begin{esimratk}
		\begin{alakohdat}
			\alakohta{
				\begin{align*}
					\frac{2x+7}{x+10} &< 1 \\
					\frac{2x+7-x-10}{x+10} &< 0 \\
					\frac{x-3}{x+10} &< 0
				\end{align*}
				$(x-3)$ ja $(x+10)$ ovat erimerkkisiä, kun $-10 < x < 3$.
			}
			\alakohta{
				\begin{align*}
					\frac{2x+7}{x+10} &> -1 \\
					\frac{2x+7+x+10}{x+10} &> 0 \\
					\frac{3x+17}{x+10} &> 0
				\end{align*}
				$(3x+17)$ ja $(x+10)$ ovat samanmerkkisiä, kun $x < -10 \;\tai x > \frac{-17}3$.
			}
			\alakohta{
				Itseisarvoepäyhtälö toteutuu, kun molemmat edellä olleet epäyhtälöt toteutuvat,
				eli kun $\frac{-17}{3} < x < 3$.
			}
		\end{alakohdat}
	\end{esimratk}
	\begin{esimvast}
		\begin{alakohdat}
			\alakohta{$-10 < x < 3$}
			\alakohta{$x < -10 \;\tai x > \frac{-17}3$}
			\alakohta{$\frac{-17}{3} < x < 3$}
		\end{alakohdat}
	\end{esimvast}
\end{esimerkki}

\begin{tehtavasivu}

\end{tehtavasivu}
