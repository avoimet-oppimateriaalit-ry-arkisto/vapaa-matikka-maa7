\chapter{Derivaatta}

% motivaatio, fysikaalinen
% sääntöjen todistamista
% aikaderivaatat, Newtonin notaatio

\section{Johdanto}

Monilla sovellusaloilla ollaan toisinaan kiinnostuneita siitä, kuinka nopeasti jokin kasvaa.
Esimerkiksi fysiikassa nopeuden muutosta mittaa suure kiihtyvyys. Taloustieteessä voidaan olla
kiinnostuneita vaikkapa kokonaisverokertymän kehityksestä suhteessa erilaisiin talouskehityksen mittareihin.

Niin fysiikan kuin taloustieteenkin mallit nojaavat matemaattiseen perustaan.
Tilanteista tehdään malleja, jotka sisältävät vapaita parametreja ja funktioita.

Isaac Newtonin ja Gottfried Wilhelm Leibnizin pioneerityö johti 1600-luvulla
matemaattisen analyysin syntyyn ja antoi työkalut funktioiden kasvunopeuden selvittämiseen.
Kasvunopeus selvitetään matematiikassa \termi{derivointi}{derivointina} tunnetulla operaatiolla.

\section{Derivaatta}

Funktion $f(x)$ \termi{derivaatta}{derivaattaa} pisteessä $a$ merkitään $f'(a)$.
Formaalisti derivaatta määritellään
\[ f'(a) = \lim\limits_{x \to a} \frac{f(x)-f(a)}{x-a} \]
tai yhtäpitävästi
\[ f'(a) = \lim\limits_{h \to 0} \frac{f(a+h)-f(a)}{h}. \]
Tämä määritelmä tunnetaan nimellä \termi{erotusosamäärän raja-arvo}{erotusosamäärän raja-arvo}.

\begin{luoKuva}{derivaatta_esimerkki}
    kuvaaja.pohja(-5, 5, -1, 8, leveys=7)
    
    kuvaaja.piirra("x**2")
    
    vari("blue")
    kuvaaja.piirra("4*x-4")
   
    vari("red")
    kuvaaja.piste((2, 4))
\end{luoKuva}

\begin{esimerkki}
	Laske funktion $f(x) = x^2$ derivaatta pisteessä $2$.
	\begin{esimratk}
		\begin{align*}
			f'(3) &= \lim\limits_{x \to 2} \frac{f(x)-f(2)}{x-2} \\
				  &= \lim\limits_{x \to 2} \frac{x^2-4}{x-2} \\
				  &= \lim\limits_{x \to 2} (x+2) \\
				  &= 4
		\end{align*}
	\end{esimratk}
	
	\begin{center}
		\naytaKuva{derivaatta_esimerkki}
	\end{center}
	
\end{esimerkki}

Funktion $f(x)$ derivaatalle pisteessä $a$ on geometrinen tulkinta:
$f'(a)$ on funktion $f(x)$ tangentin kulmakerroin pisteessä $a$.

\section{Derivaattafunktio}

On mielekästä yleistää pisteittäin määriteltyä derivaatan käsitettä.
Luonnollinen yleistys on \termi{derivaattafunktio}{derivaattafunktio}.

Funktion $f(x)$ derivaattafunktio $f'(x)$ on sellainen funktio, jolle pätee
\[ f'(x) = \lim\limits_{h \to 0} \frac{f(x+h)-f(x)}{h} \]
kaikilla funktion $f(x)$ määrittelyjoukon alkioilla $x$.

Funktiolla ei välttämättä ole derivaattafunktiota; tällöin sanomme, että funktio
ei ole \termi{derivoituvuus}{derivoituva}. Jos funktio on derivoituva, sen
derivaattafunktio on yksikäsitteinen.

\section{Derivointisäännöt I}

\laatikko[Derivointisäännöt I]{
	\begin{description}
		\item[Summan derivaatta] $(f+g)'(x) = (f'+g')(x)$
		\item[Erotuksen derivaatta] $(f-g)'(x) = (f'-g')(x)$
		\item[Vakiolla kertomisen derivaatta] $(cf)'(x) = cf'(x)$
	\end{description}
}

Derivointi on niin kutsuttu lineaarinen operaattori: derivaatta summasta on summa derivaatoista ja
vakiolla kertominen voidaan suorittaa derivoimisen ulkopuolella.

\section{Derivointisäännöt II}

Sitä vastoin tulon ja osamäärän derivointisäännöt ovat astetta monimutkaisempia.

\laatikko[Derivointisäännöt II]{
	\begin{description}
		\item[Tulon derivaatta] $(fg)'(x) = (f'g+fg')(x)$
		\item[Osamäärän derivaatta] $(\frac{f}{g})'(x) = (\frac{f'g-fg'}{g^2})(x)$
	\end{description}
}

\begin{esimerkki}
	Todista tulon derivointisääntö $(fg)'(x) = (f'g+fg')(x)$.
	\begin{esimratk}
		\begin{align*}
			(fg)'(x) &= \lim\limits_{h \to 0} \frac{(fg)(x+h)-(fg)(x)}{h} \\
					 &= \lim\limits_{h \to 0} \frac{f(x+h)g(x+h)-f(x)g(x)}{h} \\
					 &= \lim\limits_{h \to 0} \frac{f(x+h)g(x+h)+f(x+h)g(x)-f(x+h)g(x)-f(x)g(x)}{h} \\
					 &= \lim\limits_{h \to 0} (\frac{f(x+h)-f(x)}{h} g(x) + f(x+h) \frac{g(x+h)-g(x)}{h}) \\
					 &= f'(x)g(x) + f(x)g'(x) \\
					 &= (f'g)(x) + (fg')(x) \\
					 &= (f'g + fg')(x)
		\end{align*}
	\end{esimratk}
\end{esimerkki}

\section{Potenssifunktion derivaatta}

\laatikko[Potenssifunktion derivaatta]{
	$\diff{x^n} = nx^{n-1}$
}

\begin{esimerkki}
	Todista potenssifunktion derivointisääntö $\diff{x^n} = nx^{n-1}$.
	\begin{esimratk}
		Todistetaan väite induktiolla.
		\begin{description}
			\item[Alkuaskel, $n=0$] \hfill \\
			Kyseessä on vakiofunktio. Havaitaan helposti, että tällöin derivaattafunktio on identtisesti nolla.
			\item[Induktio-oletus, $n=k\geq0$] \hfill \\
			$\diff{x^k} = kx^{k-1}$.
			\item[Induktioaskel, $n=k+1$] \hfill \\
			$\diff{x^{k+1}} = \diff{x \cdot x^k} = 1 \cdot x^k + x \cdot kx^{k-1} = (k+1)x^k$. Väite on todistettu.
		\end{description}
	\end{esimratk}
\end{esimerkki}

Potenssifunktioita on helppo derivoida, joten myös polynomi- ja rationaalifunktioita on helppo derivoida.
Myöhemmillä kursseilla opitaan derivoimaan muitakin funktioita.

Derivoida voidaan useiden muuttujien suhteen. Tällöin kaikkia muita muuttujia kohdellaan vakioina.
Derivoinnin muuttuja jätetään usein merkitsemättä, jos se on asiayhteydestä selvä.

\begin{esimerkki}
	Derivoi $f(x,y,z)=x^2+xy$ muuttujan
	\begin{alakohdatrivi}
		\alakohta{$x$}
		\alakohta{$y$}
		\alakohta{$z$}
	\end{alakohdatrivi}
	suhteen. % toisaalta notaatio doo_1, doo_2, doo_3 jne. on myös aika kiva
	\begin{esimvast}
		\begin{alakohdatrivi}
			\alakohta{$2x+y$}
			\alakohta{$x$}
			\alakohta{$0$}
		\end{alakohdatrivi}
	\end{esimvast}
\end{esimerkki}

\section{Ääriarvojen määrittäminen}

Derivaattaa käyttäen voidaan ratkaista monia yksinkertaisia optimointitehtäviä.
Tätä motivoi ''ääriarvolause.''

\laatikko[''Ääriarvolause'']{
	Avoimella välillä derivoituva funktio voi saavuttaa
	lokaalin maksimin tai minimin vain sen derivaattafunktion nollakohdissa.
}

Lokaali ääriarvo on maksimi, jos toinen derivaatta pisteessä on negatiivinen ja minimi, jos
toinen derivaatta pisteessä on positiivinen. Jos toinen derivaatta on $0$, ääriarvo voi olla
maksimi, minimi tai ns. satulapiste.

\begin{esimerkki}
	Määritä funktion $f(x)=x^3-3x$ lokaalit maksimit ja minimit välillä $\aavali{0}{10}$.
	\begin{esimratk}
		Derivaattafunktioksi saadaan $f'(x)=3x^2-3$.
		\begin{align*}
			f'(x) &= 0 \\
			3x^2-3 &= 0 \\
			3x^2 &= 3 \\
			x^2 &= 1 \\
			x &= \pm 1
		\end{align*}
		Derivaattafunktion nollakohdista vain $x = 1$ sijaitsee välillä $\aavali{0}{10}$.
		Pisteessä $x = 1$ funktio saa arvon $f(1) = -2$. Toinen derivaatta on $f''(x)=6x$,
		joka saa pisteessä $x = 1$ positiivisen arvon. Siis funktiolla on lokaali minimi $-2$
		pisteessä $1$.
	\end{esimratk}
	\begin{esimvast}
		$f(1) = -2$, lokaali minimi
	\end{esimvast}
\end{esimerkki}

\begin{esimerkki}
	Pyöräilijän nopeutta yksikössä $\mathrm{\frac{m}{s}}$ mallinnetaan funktiolla % huonosti valittu malli, funktio ei fluktuoi
	\[ v(t) = -0,0000013 t^2 + 0,0052 t + 4,3, \]
	missä $t$ on aika lähdöstä sekunteina. Millä ajanhetkellä pyöräilijä saavuttaa suurimman nopeutensa, ja mikä se on?
	\begin{esimratk}
		Derivoimalla saadaan $a(t) = \dot{v}(t) = -0,0000026 t + 0,0052$.
		\begin{align*}
			a(t) &= 0 \\
			-0,0000026 t + 0,0052 &= 0 \\
			-0,0000026 t &= -0,0052 \\
			t &= 2000
		\end{align*}
		Sijoittamalla saadaan $v(2000) = 9,5$.
		Vaihdetaan yksiköt:
		\[ 2000 \; \mathrm{s} = 33 \; \mathrm{min} \; 20 \; \mathrm{s} \]
		\[ 9,5 \; \mathrm{\frac{m}{s}} =  34,2 \; \mathrm{\frac{km}{h}} \]
	\end{esimratk}
	\begin{esimvast}
		Pyöräilijä saavuttaa suurimman nopeutensa $34,2 \; \mathrm{\frac{km}{h}}$ ajettuaan $33 \; \mathrm{min} \; 20 \; \mathrm{s}$.
	\end{esimvast}
\end{esimerkki}

\begin{tehtavasivu}

\begin{tehtava}
	Laske funktioiden derivaatat pisteessä $a = 0$.
	\begin{alakohdat}
		\alakohta{$f_1(x)=x$}
		\alakohta{$f_2(x)=x^2$}
		\alakohta{$f_3(x)=x \cdot 2^x$}
		\alakohta{$f_4(x)=\frac1{x^2+1}$}
	\end{alakohdat}
	\begin{vastaus}
		\begin{alakohdat}
			\alakohta{$f'_1(0)=1$}
			\alakohta{$f'_2(0)=0$}
			\alakohta{$f'_3(0)=1$}
			\alakohta{$f'_4(0)=0$}
		\end{alakohdat}
	\end{vastaus}
\end{tehtava}

\begin{tehtava}
	Onko itseisarvofunktiolla $|x|$ derivaattaa pisteessä $x=0$?
	\begin{vastaus}
		Ei.
	\end{vastaus}
\end{tehtava}

\begin{tehtava}
	Määritä funktion $f(x)=x^4-2x^2$ lokaalit maksimit ja minimit välillä $\rr$.
	\begin{vastaus}
		\begin{itemize}
			\item $f(-1) = -1$, lokaali minimi
			\item $f(0) = 0$, lokaali maksimi
			\item $f(1) = -1$, lokaali minimi
		\end{itemize}
	\end{vastaus}
\end{tehtava}

\begin{tehtava}
	Määritä funktion $f(x)=4x+\frac1{x+4}$ lokaalit maksimit ja minimit välillä $\aavali{-4}{\infty}$.
	\begin{vastaus}
		$f(-\frac72) = -12$, lokaali minimi
	\end{vastaus}
\end{tehtava}

\begin{tehtava}
	Todista summan derivointisääntö $(f+g)'(x) = (f'+g')(x)$.
\end{tehtava}

\begin{tehtava}
	Todista osamäärän derivointisääntö $(\frac{f}{g})'(x) = (\frac{f'g-fg'}{g^2})(x)$.
\end{tehtava}

\begin{tehtava}
	Derivoi polynomit.
	\begin{alakohdat}
		\alakohta{$x^2+3x+1$}
		\alakohta{$x^5+x^3+x$}
		\alakohta{$x^3+6x^2-2x-2$}
		\alakohta{$4x^4+3x^3+2x^2+x$}
	\end{alakohdat}
	\begin{vastaus}
		\begin{alakohdat}
			\alakohta{$2x+3$}
			\alakohta{$5x^4+3x^2+1$}
			\alakohta{$3x^2+12x-2$}
			\alakohta{$16x^3+9x^2+4x+1$}
		\end{alakohdat}
	\end{vastaus}
\end{tehtava}

\begin{tehtava} % pohjustaa MAA9:aa
	Oletetaan, että tunnemme seuraavat:
	\begin{itemize}
		\item $f(x) = \sin x + 2 \cos x$, $f'(x) = -2 \sin x + \cos x$
		\item $g(x) = 5 \sin x - 2 \cos x$, $g'(x) = 2 \sin x + 5 \cos x$
	\end{itemize}
	Määritä näiden perusteella funktioiden $\sin x$ ja $\cos x$ derivaattafunktiot käyttäen tunnettuja derivaattojen laskusääntöjä. Trigonometristen funktioiden derivaattoihin syvennytään kurssissa MAA9.
	\begin{vastaus}
		\begin{itemize}
			\item $\diff{\sin x} = \cos x$
			\item $\diff{\cos x} = -\sin x$
		\end{itemize}
	\end{vastaus}
\end{tehtava}

\begin{tehtava} % haastavahko rutiininparannustehtävä
	Määritellään ns. $xD$-operaattori: funktiosta otetaan ensin derivaatta ja sen jälkeen saatu lauseke kerrotaan $x$:llä. Esimerkiksi $(xD)(x^3) = 3x^3$ ja $(xD)^2 (x^3) = 9x^3$. Laske.
	\begin{alakohdat}
		\alakohta{$(xD) (x^5+4x+1)$}
		\alakohta{$(xD) (x^6+3x^3+22x^2+10)$}
		\alakohta{$(xD)^2 (x^6+3x^3+22x^2+10)$}
		\alakohta{$(xD) (3x^6+7x^5+4x^2+43x)$}
		\alakohta{$(xD)^2 (3x^6+7x^5+4x^2+43x)$}
		\alakohta{$(xD)^2 (7x^2+32x)$}
		\alakohta{$(xD)^3 (7x^2+32x)$}
	\end{alakohdat}
	\begin{vastaus}
		\begin{alakohdat}
			\alakohta{$5x^5+4x$}
			\alakohta{$6x^6+9x^3+44x^2$}
			\alakohta{$36x^6+27x^3+88x^2$}
			\alakohta{$18x^6+35x^5+8x^2+43x$}
			\alakohta{$108x^6+175x^5+16x^2+43x$}
			\alakohta{$28x^2+32x$}
			\alakohta{$56x^2+32x$}
		\end{alakohdat}		
	\end{vastaus}
\end{tehtava}

\begin{tehtava}
	(Jatkoa edelliseen tehtävään.)
	\begin{alakohdat}
		\alakohta{$(xD) \frac1{1-x}$}
	\end{alakohdat}
\end{tehtava}

\end{tehtavasivu}
