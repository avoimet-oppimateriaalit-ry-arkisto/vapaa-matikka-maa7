\chapter{Derivaatta}

% motivaatio, fysikaalinen
% sääntöjen todistamista
% yhdistetyn funktion derivaatta vasta kurssissa MAA8

Matematiikassa funktion kasvunopeus selvitetään \termi{derivointi}{derivoinniksi} kutsutulla operaatiolla.

Funktion $f(x)$ \termi{derivaatta}{derivaattaa} pisteessä $a$ merkitään $f'(a)$.
Formaalisti derivaatta määritellään
\[ f'(a) = \lim\limits_{x \to a} \frac{f(x)-f(a)}{x-a} \]
tai yhtäpitävästi
\[ f'(a) = \lim\limits_{h \to 0} \frac{f(a+h)-f(a)}{h}. \]
Tämä määritelmä tunnetaan nimellä \termi{erotusosamäärän raja-arvo}{erotusosamäärän raja-arvo}.

\begin{esimerkki}
	Laske funktion $f(x) = x^2$ derivaatta pisteessä $3$.
	\begin{esimratk}
		\begin{align*}
			f'(3) &= \lim\limits_{x \to 3} \frac{f(x)-f(3)}{x-3} \\
				  &= \lim\limits_{x \to 3} \frac{x^2-9}{x-3} \\
				  &= \lim\limits_{x \to 3} (x+3) \\
				  &= 6
		\end{align*}
	\end{esimratk}
\end{esimerkki}

Funktion $f(x)$ derivaatalle pisteessä $a$ on geometrinen tulkinta:
$f'(a)$ on funktion $f(x)$ tangentin kulmakerroin pisteessä $a$.

Erityisen mielenkiintoisia ovat kysymykset, jotka ovat muotoa
''milloin funktio ei kasva?'' Ne vastaavat tilannetta, jossa tangentti
on vaakasuora eli funktion derivaatta on $0$.

Selvittääksemme, milloin funktion derivaatta on $0$, esittelemme aluksi
\termi{derivaattafunktio}{derivaattafunktion} käsitteen.

Funktion $f(x)$ derivaattafunktio $f'(x)$ on sellainen funktio, jolle pätee
\[ f'(x) = \lim\limits_{h \to 0} \frac{f(x+h)-f(x)}{h} \]
kaikilla funktion $f(x)$ määrittelyjoukon alkioilla $x$.

Funktiolla ei välttämättä ole derivaattafunktiota; tällöin sanomme, että funktio
ei ole \termi{derivoituvuus}{derivoituva}. Jos funktio on derivoituva, sen
derivaattafunktio on yksikäsitteinen.

\laatikko[Derivointisäännöt]{
	\begin{description}
		\item[Summan derivaatta] $(f+g)'(x) = (f'+g')(x)$
		\item[Tulon derivaatta] $(fg)'(x) = (f'g+fg')(x)$
		\item[Osamäärän derivaatta] $(\frac{f}{g})'(x) = (\frac{f'g-fg'}{g^2})(x)$
	\end{description}
}

Erikoistapauksina saadaan vielä mm.
\begin{description}
	\item[Erotuksen derivaatta] $(f-g)'(x) = (f'-g')(x)$
	\item[Vakiolla kertomisen derivaatta] $(cf)'(x) = cf'(x)$
\end{description}

\begin{esimerkki}
	Todista tulon derivointisääntö $(fg)'(x) = (f'g+fg')(x)$.
	\begin{esimratk}
		\begin{align*}
			(fg)'(x) &= \lim\limits_{h \to 0} \frac{(fg)(x+h)-(fg)(x)}{h} \\
					 &= \lim\limits_{h \to 0} \frac{f(x+h)g(x+h)-f(x)g(x)}{h} \\
					 &= \lim\limits_{h \to 0} \frac{f(x+h)g(x+h)+f(x+h)g(x)-f(x+h)g(x)-f(x)g(x)}{h} \\
					 &= \lim\limits_{h \to 0} (\frac{f(x+h)-f(x)}{h} g(x) + f(x+h) \frac{g(x+h)-g(x)}{h}) \\
					 &= f'(x)g(x) + f(x)g'(x) \\
					 &= (f'g)(x) + (fg')(x) \\
					 &= (f'g + fg')(x)
		\end{align*}
	\end{esimratk}
\end{esimerkki}

\laatikko[Potenssifunktion derivaatta]{
	$\diff{x^n} = nx^{n-1}$
}

\begin{esimerkki}
	Todista potenssifunktion derivointisääntö $\diff{x^n} = nx^{n-1}$.
	\begin{esimratk}
		Todistetaan väite induktiolla.
		\begin{description}
			\item[Alkuaskel, $n=0$] \hfill \\
			Kyseessä on vakiofunktio. Havaitaan helposti, että tällöin derivaattafunktio on identtisesti nolla.
			\item[Induktio-oletus, $n=k\geq0$] \hfill \\
			$\diff{x^k} = kx^{k-1}$.
			\item[Induktioaskel, $n=k+1$] \hfill \\
			$\diff{x^{k+1}} = \diff{x \cdot x^k} = 1 \cdot x^k + x \cdot kx^{k-1} = (k+1)x^k$. Väite on todistettu.
		\end{description}
	\end{esimratk}
\end{esimerkki}

Potenssifunktioita on helppo derivoida, joten myös polynomi- ja rationaalifunktioita on helppo derivoida.
Kursseilla MAA8 ja MAA9 tutustumme myös muiden funktioiden derivointiin.

% jotain voisi puhua Taylorin sarjojen kivuudesta?

\begin{tehtavasivu}

\begin{tehtava}
	Todista summan derivointisääntö $(f+g)'(x) = (f'+g')(x)$.
\end{tehtava}

\begin{tehtava}
	Todista osamäärän derivointisääntö $(\frac{f}{g})'(x) = (\frac{f'g-fg'}{g^2})(x)$.
\end{tehtava}

\end{tehtavasivu}
