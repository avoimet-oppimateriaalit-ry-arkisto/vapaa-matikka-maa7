\chapter{Rationaalifunktiot}

\section{Määritelmä}

\termi{rationaalifunktio}{Rationaalifunktioiksi} kutsutaan funktioita, jotka voidaan esittää kahden
polynomifunktion osamääränä.

\laatikko{
\[f(x)=\frac{P(x)}{Q(x)}\],
missä $P(x)$ ja $Q(x)$ ovat polynomifunktioita.
}

Koska $P(x)$ ja $Q(x)$ ovat polynomifunktioita, ne on määritelty kaikilla reaaliluvuilla. Tästä seuraa, että
rationaalifunktio $f(x)$ on määritelty kaikilla $x$:n reaaliarvoilla $Q$:n nollakohtia lukuunottamatta. $f(x)$ ei
ole määritelty, kun $Q(x)=0$, koska jakolaskussa nimittäjä ei voi olla nolla.

\begin{esimerkki}
	Rationaalifunktioita ovat esimerkiksi
	\begin{itemize}
		\item{$x^2 = \frac{x^2}{1}$, joka on määritelty, kun $x\in\rr$.}
		\item{$\frac1x$, joka on määritelty, kun $x\in\rr\setminus\left\{0\right\}$.}
		\item{$1+\frac2x+\frac1{x^2} = \frac{x^2+2x+1}{x^2}$, joka on määritelty, kun $x\in\rr\setminus\left\{0\right\}$.}
		\item{$\frac{x+1}{x-1}$, joka on määritelty, kun $x\in\rr\setminus\left\{1\right\}$.}
	\end{itemize}
\end{esimerkki}

Kaikki polynomit ovat määritelmän nojalla rationaalifunktioita. Lisäksi kahden rationaalifunktion summa, tulo, erotus ja osamäärä ovat aina rationaalifunktioita.

\section{Sieventäminen}

Rationaalilausekkeiden sieventäminen täytyy osata hyvin rationaaliyhtälöiden ja -epäyhtälöiden
ratkaisemiseksi.

\begin{esimerkki}
	Sievennä $2x^2 + \frac{4x}{x+1} + \frac{x^3}{x^2+2}$.
	\begin{esimratk}
		\begin{align*}
			  &2x^2 + \frac{4x}{x+1} + \frac{x^3}{x^2+2} \\
			= &\frac{2x^2 \cdot (x+1) \cdot (x^2+2)}{(x+1) \cdot (x^2+2)} + \frac{4x \cdot (x^2+2)}{(x+1) \cdot (x^2+2)} + \frac{x^3 \cdot (x+1)}{(x+1) \cdot (x^2+2)} \\
			= &\frac{2x^5 + 2x^4 + 4x^3 + 4x^2}{x^3+x^2+2x+2} + \frac{4x^3+8x}{x^3+x^2+2x+2} + \frac{x^4+x^3}{x^3+x^2+2x+2} \\
			= &\frac{2x^5+3x^4+9x^3+4x^2+8x}{x^3+x^2+2x+2}
		\end{align*}
	\end{esimratk}
\end{esimerkki}

% jokin esimerkki jossa vielä lopuksi saadaan jaettua tekijöihin

\section{Rationaaliyhtälöt}

Haluttaessa määrittää rationaalifunktion nollakohdat päädytään \termi{rationaaliyhtälö}{rationaaliyhtälöihin}.

\begin{esimerkki}
	Ratkaise yhtälöt
	\begin{alakohdat}
		\alakohta{$\frac{2x+12}{x} = 8$}
		\alakohta{$\frac{2x+12}{x} = 2$}
		\alakohta{$\frac{2x+12}{x} = -4$}
	\end{alakohdat}
	\begin{esimratk}
		Ratkaistaan yleinen tapaus $\frac{2x+12}{x} = c$, missä $c \in \rr$.
		Todetaan ensin, että $x = 0$ ei voi olla yhtälön ratkaisu. Muulloin:
		\begin{align*}
			\frac{2x+12}{x} &= c \\
			2x+12 &= cx \\
			(2-c)x+12 &= 0 \\
			(2-c)x &= -12 \\
			x &= \frac{-12}{2-c} \\
			x &= \frac{12}{c-2}
		\end{align*}
		Havaitaan, että kun $c = 2$, ratkaisuja ei ole. Muut ratkaisut saadaan sijoittamalla.
	\end{esimratk}
	\begin{esimvast}
		\begin{alakohdat}
			\alakohta{$x = 2$}
			\alakohta{ei ratkaisuja}
			\alakohta{$x = -2$}
		\end{alakohdat}
	\end{esimvast}
\end{esimerkki}

\section{Rationaaliepäyhtälöt}

\begin{esimerkki}
	Ratkaise epäyhtälöt
	\begin{alakohdat}
		\alakohta{$\frac{2x+7}{x+10} < 1$}
		\alakohta{$\frac{2x+7}{x+10} > -1$}
		\alakohta{$|\frac{2x+7}{x+10}| < 1$.}
	\end{alakohdat}
	\begin{esimratk}
		\begin{alakohdat}
			\alakohta{
				\begin{align*}
					\frac{2x+7}{x+10} &< 1 \\
					\frac{2x+7-x-10}{x+10} &< 0 \\
					\frac{x-3}{x+10} &< 0
				\end{align*}
				$(x-3)$ ja $(x+10)$ ovat erimerkkisiä, kun $-10 < x < 3$.
			}
			\alakohta{
				\begin{align*}
					\frac{2x+7}{x+10} &> -1 \\
					\frac{2x+7+x+10}{x+10} &> 0 \\
					\frac{3x+17}{x+10} &> 0
				\end{align*}
				$(3x+17)$ ja $(x+10)$ ovat samanmerkkisiä, kun $x < -10 \;\tai x > \frac{-17}3$.
			}
			\alakohta{
				Itseisarvoepäyhtälö toteutuu, kun molemmat edellä olleet epäyhtälöt toteutuvat,
				eli kun $\frac{-17}{3} < x < 3$.
			}
		\end{alakohdat}
	\end{esimratk}
	\begin{esimvast}
		\begin{alakohdat}
			\alakohta{$-10 < x < 3$}
			\alakohta{$x < -10 \;\tai x > \frac{-17}3$}
			\alakohta{$\frac{-17}{3} < x < 3$}
		\end{alakohdat}
	\end{esimvast}
\end{esimerkki}

\begin{tehtavasivu}

\begin{tehtava}
	Ratkaise epäyhtälö $|\frac{x^2+1}{2x+5}| < \frac15$
	\begin{vastaus}
		$0 < x < \frac25$
	\end{vastaus}
\end{tehtava}

\begin{tehtava}
% Laatinut Sampo Tiensuu 2014-01-09
Sievennä seuraavat rationaalilausekkeet:
\begin{alakohdat}
    \alakohta{$x+\frac{x}{1}+\frac{1}{x}$}
    \alakohta{$1:(x+1)+1$}
    \alakohta{$-\frac{7+x^2}{5}-\frac{2+x}{x}$}
    \alakohta{$\frac{2x+2x}{5x^2+\frac{x}{x+1}}$}
%    \alakohta{$\frac{ax+by}{cx^2+\frac{x}{x+1}}$}
    \alakohta{$2x/5-3/5$}			
\end{alakohdat}

\begin{vastaus}
\begin{alakohdat}
    \alakohta{$\frac{2x^2+1}{x}$}
    \alakohta{$\frac{x+2}{x+1}$}
    \alakohta{$\frac{x^3-13x-10}{5x}$}
    \alakohta{$\frac{4x^2+4}{5x^2+5x+1}$}
    \alakohta{$\frac{2x-3}{5}$}
\end{alakohdat}
\end{vastaus}
\end{tehtava}

\begin{tehtava}
Kahden polynomin summa, erotus ja tulo ovat aina polynomeja. Todista tähän vedoten, että kahden rationaalifunktion summa, tulo, erotus ja osamäärä ovat aina rationaalifunktioita.
\begin{vastaus}
Vinkki: Lavenna
\end{vastaus}
\end{tehtava}

\end{tehtavasivu}
